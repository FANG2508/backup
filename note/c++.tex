%!TEX program = xelatex
\documentclass[cn,hazy,blue,screen,14pt]{note}
\usepackage{tikz}
\title{学习笔记}

\author{}

%\version{0.00}
\date{\zhtoday}

\begin{document}

\maketitle
\newpage
\tableofcontents
\newpage

\noindent lambda表达式: [captures] (params) -> return type $\{$Statments;$\}$


[] 不截取任何变量

[\&] 截取外部作用域中所有变量,并作为引用在函数体中使用

[=] 截取外部作用域中所有变量,并拷贝一份在函数体中使用

[=, \&foo] 截取外部作用域中所有变量,并拷贝一份在函数体中使用,但是对foo变量使用引用

[bar] 截取bar变量并且拷贝一份在函数体重使用,同时不截取其他变量

[this] 截取当前类中的this指针。如果已经使用了\&或者=就默认添加此选项。
\\\\
不能通过类型名来显式声明对应的对象,但可以利用auto关键字和类型推导:

auto f=[](int a,int b)$\{$return a>b;$\}$;


泛型lambda (C++14):[](const auto\& a,const auto\& b)$\{$return a>b;$\}$

\newpage

\noindent transform: 对容器给定范围内元素进行操作

一元: transform(iterator begin, iterator end, iterator output.begin, operation) //operation可以是函数指针、函数对象或lambda表达式

二元: transform(iterator begin, iterator end, iterator begin2, iterator output.begin, operation) //第二个容器取的end取决于第一个容器的元素个数

\end{document}

